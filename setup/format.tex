% !Mode:: "TeX:UTF-8"

%%%%%%%%%%%%%%%%% Fonts Definition and Basics %%%%%%%%%%%%%%%%%
\newcommand{\song}{\CJKfamily{song}}    % 宋体
\newcommand{\fs}{\CJKfamily{fs}}        % 仿宋体
\newcommand{\kai}{\CJKfamily{kai}}      % 楷体
\newcommand{\hei}{\CJKfamily{hei}}      % 黑体
\newcommand{\li}{\CJKfamily{li}}        % 隶书
\newcommand{\chuhao}{\fontsize{28pt}{28pt}\selectfont}       % 初号, 单倍行距
\newcommand{\yihao}{\fontsize{26pt}{26pt}\selectfont}       % 一号, 单倍行距
\newcommand{\xiaoyi}{\fontsize{24pt}{24pt}\selectfont}      % 小一, 单倍行距
\newcommand{\erhao}{\fontsize{22pt}{1.25\baselineskip}\selectfont}       % 二号, 1.25倍行距
\newcommand{\xiaoer}{\fontsize{18pt}{18pt}\selectfont}      % 小二, 单倍行距
\newcommand{\sanhao}{\fontsize{16pt}{16pt}\selectfont}      % 三号, 单倍行距
\newcommand{\xiaosan}{\fontsize{15pt}{15pt}\selectfont}     % 小三, 单倍行距
\newcommand{\sihao}{\fontsize{14pt}{14pt}\selectfont}       % 四号, 单倍行距
\newcommand{\xiaosi}{\fontsize{12pt}{12pt}\selectfont}      % 小四, 单倍行距
\newcommand{\wuhao}{\fontsize{10.5pt}{10.5pt}\selectfont}   % 五号, 单倍行距
\newcommand{\xiaowu}{\fontsize{9pt}{9pt}\selectfont}        % 小五, 单倍行距

% 重新定义了波浪符~的意义
\CJKtilde

% 定义章的pre-post名称
\newcommand\prechaptername{第}
\newcommand\postchaptername{章}

% 调整中文字符的表示,行内占一个字符宽度,行尾占半个字符宽度
% 行末半角?
\punctstyle{hangmobanjiao}             

% 调整罗列环境的布局
\setitemize{leftmargin=3em,itemsep=0em,partopsep=0em,parsep=0em,topsep=-0em}
\setenumerate{leftmargin=3em,itemsep=0em,partopsep=0em,parsep=0em,topsep=0em}

% 避免宏包 hyperref 和 arydshln 不兼容带来的目录链接失效的问题。
\def\temp{\relax}
\let\temp\addcontentsline
\gdef\addcontentsline{\phantomsection\temp}

% 自定义项目列表标签及格式 \begin{publist} 列表项 \end{publist}
\newcounter{pubctr} %自定义新计数器
\newenvironment{publist}{%%%%%定义新环境
\begin{list}{[\arabic{pubctr}]} %%标签格式
    {
     \usecounter{pubctr}
     \setlength{\leftmargin}{2.5em}   % 左边界 \leftmargin =\itemindent + \labelwidth + \labelsep
     \setlength{\itemindent}{0em}     % 标号缩进量
     \setlength{\labelsep}{1em}       % 标号和列表项之间的距离,默认0.5em
     \setlength{\rightmargin}{0em}    % 右边界
     \setlength{\topsep}{0ex}         % 列表到上下文的垂直距离
     \setlength{\parsep}{0ex}         % 段落间距
     \setlength{\itemsep}{0ex}        % 标签间距
     \setlength{\listparindent}{0pt}  % 段落缩进量
    }}
{\end{list}}

\makeatletter
	\renewcommand\normalsize{
		\@setfontsize\normalsize{12pt}{12pt} % 小四对应 12 pt
		\setlength\abovedisplayskip{4pt}
		\setlength\abovedisplayshortskip{4pt}
		\setlength\belowdisplayskip{\abovedisplayskip}
		\setlength\belowdisplayshortskip{\abovedisplayshortskip}
		\let\@listi\@listI}
	
	% 不同的行距设置
	% TJU原始值1.63
	% 设为1.8则一页31行,1.95则一页29行(目前采用值)
	\def\defaultfont{\renewcommand{\baselinestretch}{1.95}\normalsize\selectfont} % 设置行距,正文一页29行
	
	% 控制字间距,使每行 34 个汉字
	\renewcommand{\CJKglue}{\hskip -0.1 pt plus 0.08\baselineskip} 
\makeatother

%%%%%%%%%%%%% Contents 目录 %%%%%%%%%%%%%%%%%

\renewcommand{\contentsname}{目\qquad 录}

% 控制目录深度,改为1
\setcounter{tocdepth}{1}

\titlecontents{chapter}[2em]{\vspace{.0\baselineskip}\sihao\song}	% 可以重调skip
	{\prechaptername~\thecontentslabel~\postchaptername\quad}{}
	{\!\titlerule*[5pt]{$\cdot$}\!\!\!\!\sihao\contentspage}	% 调整点的距离
\titlecontents{section}[3em]{\vspace{-0.1\baselineskip}\xiaosi\song}
	{\thecontentslabel\quad}{}
	{\!\titlerule*[5pt]{$\cdot$}\!\!\!\!\xiaosi\contentspage}
\titlecontents{subsection}[4em]{\vspace{-0.2\baselineskip}\wuhao\song}
	{\thecontentslabel\quad}{}
	{\!\titlerule*[5pt]{$\cdot$}\!\!\!\!\wuhao\contentspage}
             
%%%%%%%%%% Chapter and Section 章节 %%%%%%%%%%%%%

\setcounter{secnumdepth}{4}
\setlength{\parindent}{2em}

% 如果使用第“一”章
%\renewcommand{\chaptername}{\prechaptername\CJKnumber{\thechapter}\postchaptername}
% 使用第“1”章
\renewcommand{\chaptername}{\prechaptername~\thechapter~\postchaptername}

% 此处修改的chapter title会被主文件定义覆盖
% chapter标题格式:小二,黑体,居中
\titleformat{\chapter}{\centering\xiaoer\hei}{\chaptername}{2em}{}
\titlespacing{\chapter}{0pt}{0.1\baselineskip}{0.8\baselineskip}

% section标题格式:小三,宋体加粗,左对齐
\titleformat{\section}{\xiaosan\song\bfseries}{\thesection}{1em}{}
\titlespacing{\section}{0pt}{0.15\baselineskip}{0.25\baselineskip}

% subsection标题格式:四号,宋体加粗,左对齐
\titleformat{\subsection}{\sihao\song\bfseries}{\thesubsection}{1em}{}
\titlespacing{\subsection}{0pt}{0.1\baselineskip}{0.3\baselineskip}

% subsubsection标题格式:小四,宋体加粗,左对齐
\titleformat{\subsubsection}{\xiaosi\song\bfseries}{\thesubsubsection}{1em}{}
\titlespacing{\subsubsection}{0pt}{0.05\baselineskip}{0.1\baselineskip}

%%%%%%%%%% Table, Figure and Equation 图/表/公式 %%%%%%%%%%%%%%%%%

\renewcommand{\tablename}{表}
\renewcommand{\figurename}{图}

% 使图编号为 7-1 的格式
\renewcommand{\thefigure}{\arabic{chapter}-\arabic{figure}}

% 使子图编号为 a) 的格式
%\renewcommand{\thesubfigure}{\alph{subfigure})}
% 使子图编号为 (a) 的格式
\renewcommand{\thesubfigure}{(\alph{subfigure})}

% 使子表编号为 (a) 的格式
\renewcommand{\thesubtable}{(\alph{subtable})}
% 使表编号为 7-1 的格式
\renewcommand{\thetable}{\arabic{chapter}-\arabic{table}}
% 使公式编号为 7-1 的格式
\renewcommand{\theequation}{\arabic{chapter}-\arabic{equation}}

\makeatletter
	% 使子图引用也是7-1a)或7-1(a)的形式
	\renewcommand{\p@subfigure}{\thefigure}
\makeatother

% 定制浮动图形和表格标题样式
\makeatletter
	\long\def\@makecaption#1#2{
	   \vskip\abovecaptionskip
	   \sbox\@tempboxa{\centering\wuhao\song{#1\quad #2}}
	   \ifdim \wd\@tempboxa >\hsize
	     \centering\wuhao\song{#1\quad #2} \par	% narrower
	   \else
	     \global \@minipagefalse
	     \hb@xt@\hsize{\hfil\box\@tempboxa\hfil}
	   \fi
	   \vskip\belowcaptionskip}
\makeatother

% 用来控制longtable表头分隔符
\captiondelim{~~~~} 

%%%%%%%%%% Theorem Environment 定理 %%%%%%%%%%%%%%%%%
\theoremstyle{plain}
\theorembodyfont{\song\rmfamily}
\theoremheaderfont{\hei\rmfamily}
\newtheorem{theorem}{定理~}[chapter]
\newtheorem{lemma}{引理~}[chapter]
\newtheorem{axiom}{公理~}[chapter]
\newtheorem{proposition}{命题~}[chapter]
\newtheorem{prop}{性质~}[chapter]
\newtheorem{corollary}{推论~}[chapter]
\newtheorem{conclusion}{结论~}[chapter]
\newtheorem{definition}{定义~}[chapter]
\newtheorem{conjecture}{猜想~}[chapter]
\newtheorem{example}{例~}[chapter]
\newtheorem{remark}{注~}[chapter]
%\newtheorem{algorithm}{算法~}[chapter]
\newenvironment{proof}{\noindent{\hei 证明:}}{\hfill $ \square $ \vskip 4mm}
\theoremsymbol{$\square$}

%%%%%%%%%% Page: number, header and footer 页面设置 %%%%%%%%%%%%%%%%%

%\frontmatter 或 \pagenumbering{roman}
%\mainmatter 或 \pagenumbering{arabic}
\makeatletter
	\renewcommand\frontmatter{\clearpage
		\@mainmatterfalse}
\makeatother

%%%%%%%%%%%% References 参考文献 %%%%%%%%%%%%%%%%%

\renewcommand{\bibname}{参考文献}
% 重定义参考文献样式,来自thu
\makeatletter
\renewenvironment{thebibliography}[1]{
    %\titleformat{\chapter}{\raggedright\sihao\hei}{\chaptername}{2em}{}
    %\titleformat{\chapter}{\centering\sihao\hei}{\chaptername}{2em}{}
    %\titleformat{\chapter}{\centering\xiaoer\hei}{\chaptername}{2em}{}
   \chapter*{\bibname}
   \wuhao
   \list{\@biblabel{\@arabic\c@enumiv}}
        {\renewcommand{\makelabel}[1]{##1\hfill}
         \settowidth\labelwidth{0 cm}
         \setlength{\labelsep}{0pt}
         \setlength{\itemindent}{0pt}
         \setlength{\leftmargin}{\labelwidth+\labelsep}
         \addtolength{\itemsep}{-0.7em}
%         \addtolength{\itemsep}{-1.0em}
         \linespread{1.5}\selectfont	% 调整每个参考文献项内的间距 !!!
         \usecounter{enumiv}
         \let\p@enumiv\@empty
         \renewcommand\theenumiv{\@arabic\c@enumiv}}
    \sloppy\frenchspacing
    \clubpenalty4000
    \@clubpenalty \clubpenalty
    \widowpenalty4000
    \interlinepenalty4000
    \sfcode`\.\@m}
   {\def\@noitemerr
     {\@latex@warning{Empty `thebibliography' environment}}
    \endlist\frenchspacing}
\makeatother

% 缩小参考文献间的垂直间距
\addtolength{\bibsep}{-0.5em}

% 每个条目自第二行起缩进的距离
\setlength{\bibhang}{2em}

% 参考文献引用作为上标出现
\makeatletter
	\def\@cite#1#2{\textsuperscript{[{#1\if@tempswa , #2\fi}]}}
\makeatother

% 引用格式
\bibpunct{[}{]}{,}{s}{}{,}

%%%%%%%%%%%% Cover 封面、摘要、版权、致谢格式定义 %%%%%%%%%%%%%%%%%
 
\makeatletter % 一直到结尾

\def\ctitle#1{\def\@ctitle{#1}}\def\@ctitle{}
\def\etitle#1{\def\@etitle{#1}}\def\@etitle{}
\def\csubject#1{\def\@csubject{#1}}\def\@csubject{}
\def\esubject#1{\def\@esubject{#1}}\def\@esubject{}
\def\cauthor#1{\def\@cauthor{#1}}\def\@cauthor{}
\def\eauthor#1{\def\@eauthor{#1}}\def\@eauthor{}
\def\csupervisor#1{\def\@csupervisor{#1}}\def\@csupervisor{}
\def\esupervisor#1{\def\@esupervisor{#1}}\def\@esupervisor{}
\def\cdate#1{\def\@cdate{#1}}\def\@cdate{}
\long\def\cabstract#1{\long\def\@cabstract{#1}}\long\def\@cabstract{}
\long\def\eabstract#1{\long\def\@eabstract{#1}}\long\def\@eabstract{}
\def\ckeywords#1{\def\@ckeywords{#1}}\def\@ckeywords{}
\def\ekeywords#1{\def\@ekeywords{#1}}\def\@ekeywords{}
\def\cheading#1{\def\@cheading{#1}}\def\@cheading{}

\pagestyle{fancy}
  \fancyhf{}
  %\fancyhead[C]{\song\wuhao \@cheading}  % 页眉
  \lhead{\song\wuhao \@cheading}  % 左页眉
%  \rhead{\prechaptername\CJKnumber{\thechapter}\postchaptername}    % 右页眉
  \rhead{\prechaptername~\thechapter~\postchaptername}    % 右页眉
  \fancyfoot[C]{\song\xiaowu ~\thepage~}
\newlength{\@title@width}

% 定义封面
\def\makecover{
   \phantomsection
    \pdfbookmark[-1]{\@ctitle}{ctitle}

\begin{titlepage}
\vspace*{31.5pt}
\begin{center}

  \vspace*{21pt}
  \hei\chuhao{\textbf{中山大学硕士学位论文}}

  \vspace*{60pt}
  \song\xiaoer{\@ctitle}

  \xiaoer{\textrm{\@etitle}}
 
   \vspace*{80pt}
   \setlength{\@title@width}{6cm}	% 控制封面中下划线的长度。
   {\sihao\song{
   \begin{tabular}{lc}
     学~~位~~申~~请~~人   &  \underline{\makebox[\@title@width][c]{\@cauthor}} \\
     导师姓名及职称       &  \underline{\makebox[\@title@width][c]{\@csupervisor}} \\
     专~~~~业~~~~名~~~~称 &  \underline{\makebox[\@title@width][c]{\@csubject}}\\
   \end{tabular}}
  }
 
  \vspace*{42pt}
   \setlength{\@title@width}{5cm}
   {\sanhao\song{
   \begin{tabular}{lc}
     答辩委员会主席(签名):  &  \underline{\makebox[\@title@width][c]{~}} \\
     答辩委员会委员(签名):  &  \underline{\makebox[\@title@width][c]{~}} \\
     ~ &  \underline{\makebox[\@title@width][c]{~}}\\
     ~ &  \underline{\makebox[\@title@width][c]{~}}\\
     ~ &  \underline{\makebox[\@title@width][c]{~}}\\
     ~ &  \underline{\makebox[\@title@width][c]{~}}\\
   \end{tabular}}	% 不加粗
  }

 \vspace*{60pt}

  \vspace*{21pt}

\song\sanhao{\textbf{\@cdate}}
\end{center}
\end{titlepage}

% 空白页
%\newpage
%\thispagestyle{empty}
%\mbox{}

%%%%%%%%%%%%%%%%%%%   Originality Statement  %%%%%%%%%%%%%%%%%%%%%%%
\clearpage
\pdfbookmark[0]{论文原创性声明}{originality}
\chapter*{\centering\sanhao\song\bfseries 论文原创性声明}
\song\defaultfont
本人郑重声明:所呈交的学位论文,是本人在导师的指导下,独立进行研究工作所取得的成果。除文中已经注明引用的内容外,本论文不包含任何其他个人或集体已经发表或撰写过的作品成果。对本文的研究作出重要贡献的个人和集体,均已在文中以明确方式标明。本人完全意识到本声明的法律结果由本人承担。

\vspace*{40pt}
\begin{flushright}
\setlength{\@title@width}{5cm}
  {\sihao\song{
  \begin{tabular}{lc}
    学位论文作者签名:           &  \underline{\makebox[\@title@width][c]{~}} \\
    \qquad\qquad\qquad 日~~期:  &  \underline{\makebox[\@title@width][c]{~}} \\
  \end{tabular}}
 }
\end{flushright}

%%%%%%%%%%%%%%%%%%%   Authorization Statement  %%%%%%%%%%%%%%%%%%%%%%%
\vspace*{60pt}
\pdfbookmark[0]{学位论文使用授权声明}{authorization}
\begin{center}
  \sanhao\song\bfseries{学位论文使用授权声明}
\end{center}

\song\defaultfont
本人完全了解中山大学有关保留、使用学位论文的规定,即:学校有权保留学位论文并向国家主管部门或其指定机构送交论文的电子版和纸质版,有权将学位论文用于非赢利目的的少量复制并允许论文进入学校图书馆、院系资料室被查阅,有权将学位论文的内容编入有关数据库进行检索,可以采用复印、缩印或其他方法保存学位论文。

\vspace*{40pt}
\begin{flushright}
\setlength{\@title@width}{5cm}
  {\sihao\song{
  \begin{tabular}{ll}
    学位论文作者签名: \qquad\qquad\qquad  &  导师签名: \qquad\qquad\qquad\\
    日期: \qquad 年\qquad 月\qquad 日     &  日期: \qquad 年\qquad 月\qquad 日 \\
  \end{tabular}}
 }
\end{flushright}
\thispagestyle{empty}   % 去掉页码

% 空白页
%\newpage
%\thispagestyle{empty}
%\mbox{}

%%%%%%%%%%%%%%%%%%% Abstract and Keywords 摘要和关键词 %%%%%%%%%%%%%%%%%%%%%%%

%中文摘要格式
\clearpage
\markboth{摘~要}{摘~要}
\pdfbookmark[0]{摘~~要}{cabstract}

% 摘要不加到目录中
%\addcontentsline{toc}{chapter}{摘要}

% 开始罗马数字编号
\setcounter{page}{1}
\pagenumbering{Roman}
\thispagestyle{plain}

\begin{flushleft}
\setlength{\@title@width}{5cm}
  {\xiaosi\song{
	\begin{tabular}{lcl}
	论文题目 & : & \@ctitle\\
	专业 & : & \@csubject \\
	硕士生 & : & \@cauthor \\
	指导老师 & : & \@csupervisor \\
	\end{tabular}}
 }
\end{flushleft}

% 中文摘要:小二,黑体加粗,居中
\begin{center}
\xiaoer\hei\bfseries 摘\qquad 要
\end{center}

\vspace{\baselineskip} % 新增摘要后空行

% 插入中文摘要
\song\defaultfont
\@cabstract
\vspace{\baselineskip}

\hangafter=1\hangindent=52.3pt\noindent
{\hei\xiaosi 关键词:} \@ckeywords
%\thispagestyle{empty}

% 英文摘要格式
\clearpage
\markboth{ABSTRACT}{ABSTRACT}
\pdfbookmark[0]{ABSTRACT}{eabstract}

% 摘要不加到目录中
%\addcontentsline{toc}{chapter}{ABSTRACT}

\thispagestyle{plain}

% 如果英文title太长,手动分成两行
\begin{flushleft}
\setlength{\@title@width}{5cm}
  {\xiaosi{
  \begin{tabular}{ll}
	  Title:      &  \@etitle \\
    %Title: & If the title is too long, break it into two lines manually \\
    %& the longer part \\
    Major:      &  \@esubject \\
    Name:       &  \@eauthor \\
    Supervisor: &  \@esupervisor \\
  \end{tabular}}
 }
\end{flushleft}

% ABSTRACT三号居中
\begin{center}
\sanhao{\bf{ABSTRACT}}
\end{center}
\vspace{\baselineskip}

% 插入英文摘要
\@eabstract
\vspace{\baselineskip}

\hangafter=1\hangindent=60pt\noindent
{\textbf{Keywords:}} \@ekeywords
\thispagestyle{plain}

}
\makeatother
